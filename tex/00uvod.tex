Problém obchodného cestujúceho je pomerne známy a dobre preskúmaný problém
v teoretickej informatike. Navyše má veľa aplikácii v plánovaní, logistike, atď.

V tejto práci sa zaoberáme modifikáciou problému obchodného cestujúceho.
Miesto jednej hamiltonovskej kružnice budeme hľadať dve hranovo disjuktné
hamiltonovské kružnice a budeme minimalizovať ich celkovú dĺžku.
Tento problém nazveme problémom dvoch obchodných cestujúcich.

Ako prvý tento problém zaviedol \cite{krarup}. V literatúre sa spomína niekoľko aplikácii
tohoto problému. Napríklad \cite{appl} skúma návrh ciest pre strážnikov, kde je cieľom
nájsť niekoľko rôznych kružníc, aby sa cesty neopakovali. Ďalšie aplikácie
spomína \cite{duchenne}. Napríklad pri transporte nebezpečných materiálov
chceme distribuovať riziko a trasy kamiónov navrhnúť tak, aby zdieľali čo
najmenej ciest.

V tejto práci navrhneme niekoľko nových algoritmov pre riešenie
problému dvoch obchodných cestujúcich. Práca je členená nasledovne:
V kapitole 1 formálne zadefinujeme problém a jeho variácie, ktoré budeme riešiť.
V kapitole 2 zhrnieme doterajšie výsledky pri skúmaní problému dvoch
obchodných cestujúcich. V kapitolách 3 a 4 sme navrhli algoritmy, ktorých
význam je hlavne teoretický. V kapitole 3 sme navrhli exaktný algoritmus
pre riešenie problému dvoch obchodných cestujúcich. V kapitole 4 popisujeme
PTAS. Napokon v kapitole 5 navrhneme niekoľko nových heuristík
porovnávame ich výsledky s heuristikami známymi doteraz. 


