Väčšina programovacích jazykov okrem základných príkazov obsahuje aj sadu knižníc,
ktorá implementuje ďalšie užitočné funkcionality. Pre nás sú zaujímavé hlavne knižnice, ktoré
implementujú základné algoritmy a kontajnery v danom jazyku (príkladom je Standard Template
Library - \cite{STL} z C++). Dôležitou vlastnosťou týchto algoritmov je ich schopnosť
fungovať nezávisle na type spracovávaných dát.

V kompilátoroch pre jazyk Pascal sa doteraz takéto knižnice nevyskytovali.
Jedným z hlavných dôvodov bola slabá alebo takmer žiadna podpora pre generics.
V roku 2008 vyšla verzia 2.2 kompilátora FreePascal, ktorá zaviedla chabú podporu pre
generics. My sme sa rozhodli niektoré základné kontejnery pre FreePascal pomocou
generics implementovať a pretlačiť ich medzi knižnice distribuované spolu s FreePascalom.

Členenie tejto práce je nasledovné. V prvej kapitole prezentujeme rozhranie
našej knižnice, spolu s niekoľkými jednoduchými príkladmi použitia. Táto
kapitola tvorí akýsi základný manuál. V druhej kapitole prezentujeme implementačné
detaily jednotlivých komponentov. 
V poslednej kapitole ukážeme niekoľko experimentálnych porovnaní efektívnosti
našej implementácie s implementáciou STL v C++.
