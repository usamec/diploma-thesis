Zamerajme sa teraz na inštancie, kde vrcholy grafu reprezentujú body v rovine a hrany
vzdialenosti medzi nimi. V prípade obyčajného problému obchodného cestujúceho
existuje PTAS (polynomial time approximation scheme) (\cite{Arora}), t.j. algoritmus, ktorý
pre ľubovoľný aproximačný pomer beží v polynomiálnom čase od veľkosti vstupu.

My v nasledujúcej časti použijeme myšlienky z tohoto algoritmu a ukážeme ako spraviť PTAS pre
problém dvoch obchodných cestujúcich.

Hlavnou myšlienkou algoritmu pomocou techniky rozdeľuj a panuj vybudovať nad vstupom
randomizovaný quad-tree. Následne ukážeme, že pre $(1 + 1/c)$-aproximačný algoritmus stačí, aby
TSP-cesty pretínali hrany delenia iba $O(c)$ krát. Navyše tieto prieniky sa môžu
udiať iba v niektorých špeciálnych bodoch.

\section{Rekurzívne delenie}

%Rovinu v ktorej ležia vstupné body budeme rekurzívne deliť na menšie časti.
%Najprv rovinu rozdelíme vodorovnou a zvislou čiarou na štyri časti, každú z nich na ďalšie štyri, ...
%Takto by sme dostali tzv. quadtree. Na správnu fukčnosť algoritmu potrebujeme ale tzv.
%posunutý quadtree, kde deliace čiary neprechádzajú stredom, ale sú vyberané náhodne.
%

Pre začiatok predpokladajme, že všetky body majú celočíselné súradnice a vzdialenosť
medzi nimi je aspoň $8$. Neskôr ukážeme, že každý vstup vieme takto upraviť a ak budeme mať PTAS
pre upravený vstup, tak budeme mať PTAS aj pre pôvodný vstup.

Nech všetky body ležia vo vnútri štvorca $S$ so stranou $L$, ktorého ľavý dolný roh
má súradnice $[0,0]$. Navyše bez ujmy na všeobecnosti nech je $L$ mocnina dvojky.

{\bf Rekurzívnym delením} štvorca $S$ nazývame jeho rekurzívne delenie na menšie štvorce tak, že
každy štvorec rozdelíme {\bf deliacimi čiarami} na štyri rovnaké štvorce.
Deliť prestaneme, keď je strana štvorca $\leq 1$. Toto delenie predstavuje
$4$-árny strom, kde synovia každé ho štvorca sú štyri menšie štvorce. Hĺbka takéhoto stromu je
najviac $O(\lg L)$. 
Štvorcom delenia intuitívne priradíme úrovne. Štvorec $S$ má úroveň $0$, jeho synovia úroveň $1$,
atď. Deliaca čiara má úroveň $i$ vtedy, keď obsahuje hranu nejakého štvorca úrovne $i$, ale žiadneho
úrovne väčšej ako $i$.

{\bf Quadtree} definujeme podobne 
až na to, že rekurzívne delenie skončí vtedy, keď v danom štvorci je najviac jeden bod. Keďže každý
list quadtree obsahuje bod zo vstupu, alebo je súrodenec takého listu, tak quadtree má maximálne
$O(n)$ listov a teda dokopy najviac $O(n \lg L)$ štvorcov.

\smallskip

Teraz popíšeme posunuté delenie. Nech $a, b$ sú celé čísla a platí $0 \leq a, b < L$.
$(a,b)$-posunutie rekurzívneho delenia definujeme tak, že $x$-, $y-$ súradnice každej deliacej čiary
zvýšime o $a$ resp. $b$ a následne spočítame ich zvyšok po delení číslom $L$ (vstupnými bodmi
nehýbeme). Takisto posunieme aj okrajové čiary štvorca (začnú tvoriť deliace čiary). Pôvodný štvorec
pokladáme za zacyklený, teda okrajové oblasti pri posunutom delení tvoria súvislú oblasť ako na
obrázku.

TODO obrázok

Z posunutého rekurzívneho delenia vieme dodefinovať quadtree s posunom tak, že nebudeme deliť
štvorce, ktoré obsahujú najviac jeden bod.

\section{Portály a $(m,r)$-ľahké cesty}

Nás algoritmus dovolí TSP-cestám pretínať čiary delenia iba v niekoľkých
predpísaných bodoch, tzv. {\bf portáloch}. Toto je na prvý pohľad trochu kontraintuitívne.
My ale dovolíme cestujúcim cestovať medzi bodmi po ľubovoľnej čiare, teda nie nutne najkratšej
ceste medzi dvoma bodmi (viď. obrázok nižšie). Stále ale zachovávame disjuktnosť ciest, t.j. keď
jeden cestujúci ide z vrchola $x$ do vrcholu $y$, tak druhý nemôže ísť z $x$ do $y$ ani z $y$ do
$x$ (nech ide ľubovoľne zakrivenou cestou). 

\begin{definicia}
Nech $m$ je párne kladné celé číslo. $m$-regulárna množina portálov pre posunuté rekurzívne
delenie je množina bodov taká, že každá deliacia čiara úrovne $i$ obsahuje portály vo
vzdialenostiach $\frac{L}{m2^i}$. 
\end{definicia}

TODO obrázok

\begin{definicia}
TSP-cesta sa nazýva $(m,r)$-ľahká ak pretína čiary rekurzívneho delenia iba v $m$-regulárnej
množine portálov a navyše každú hranu štvorca pretína najviac $r$-krát.
\end{definicia}


