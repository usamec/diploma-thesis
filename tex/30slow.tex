V tejto kapitole sa budeme venovať pomalým metódam riešenia problému
dvoch obchodných cestujúcich.

\section{Skúšanie všetkých možností}

V grafe s $n$ vrcholmi sa nachádza $n!$ rôznych kružníc. Pokiaľ hľadáme
dve disjuktné kružnice, tak rôzných dvojíc kružníc je $\tilde{O}((n!)^2)$.
Každú z týchto možností vieme overiť v čase $O(n)$. Celkový čas algoritmu
je teda $\tilde{O}((n!)^2)$.

\section{Dynamické programovanie pre problém obchodného cestujúceho}

Problém obchodného cestujúceho sa dá riešiť pomocou dynamického
programovania v čase $O(2^n n^2)$ (\cite{Held}). Hlavnou myšlienkou
tohoto algoritmu je riešiť podproblémy tvaru: Nájdite najkratšiu cestu, ktorá
začína vo vrchole 1, prechádza cez množinu vrcholov $X \subseteq V$ a končí vo vrchole
$x \in X$.

Tento algoritmus vieme využiť pomerne priamočiaro. Pre každú z $n!$ možných kružníc
v čase $\tilde{O}(2^n)$ nájdeme najkratšiu druhú kružnicu. Takto dostáneme
algoritmus, ktorý beží v čase $\tilde{O}(2^n n!)$. 

