V tejto kapitole sa budeme venovať pomalým metódam riešenia problému
dvoch obchodných cestujúcich.

\section{Skúšanie všetkých možností}

V grafe s $n$ vrcholmi sa nachádza $n!$ rôznych kružníc. Pokiaľ hľadáme
dve disjuktné kružnice, tak rôzných dvojíc kružníc je $\tilde{O}((n!)^2)$.
Každú z týchto možností vieme overiť v čase $O(n)$. Celkový čas algoritmu
je teda $\tilde{O}((n!)^2)$.

\section{Dynamické programovanie pre problém obchodného cestujúceho}

Problém obchodného cestujúceho sa dá riešiť pomocou dynamického
programovania v čase $O(2^n n^2)$ (\cite{Held}). Hlavnou myšlienkou
tohoto algoritmu je riešiť podproblémy tvaru: Nájdite najkratšiu cestu, ktorá
začína vo vrchole 1, prechádza cez množinu vrcholov $X \subseteq V$ a končí vo vrchole
$x \in X$.

Tento algoritmus vieme využiť pomerne priamočiaro. Pre každú z $n!$ možných kružníc
v čase $\tilde{O}(2^n)$ nájdeme najkratšiu druhú kružnicu. Takto dostaneme
algoritmus, ktorý beží v čase $\tilde{O}(2^n n!)$. 

\section{Lepšie dynamické programovanie}

V tomto prípade budeme robiť dynamické programovanie priamo pre problém
dvoch obchodných cestujúcich. Náš prístup bude podobný ako pri probléme
obchodného cestujúceho. 
Budeme hľadať jednu najkratšiu cestu cez danú podmnožinu vrcholov
$X$, ktorá začína vo vrchole 1 a končí v danom vrchole $x$. Potrebujeme ešte vyriešiť
druhú kružnicu. V tomto prípade použijeme pomerne hrubú silu.
Množinu $X$ rozložíme na tri disjuktné množiny $A, B, C$.
Vrcholy v množine $A$ zatiaľ do druhej kružnice zapojené nebudú. Vrcholy v množine
$C$ budú spojené v druhej kružnici s inými dvoma vrcholmi.
A vrcholy v množine $B$ budú spojené iba s jedným vrcholom. Čiže druhá kružnica je
rozložená na niekoľko ciest, ktoré začínajú vo vrcholoch množiny $B$ idú cez niekoľko
(alebo nula) vrcholov z množiny $C$ a následne končia vo vrchole z množiny $B$.
Ešte ale potrebujeme jednu informáciu navyše -- ktoré koncové body ciest v množine
$B$ sú spojené (čiže máme rozdelenie množiny $B$ do dvojíc).

A v konečnom dôsledku pre každú kombináciu množín $A, B, C$ vrchola $x$ a rozdelení
vrcholov v množine $B$ hľadáme cestu $p$ cez vrcholy v množine $A \cup B \cup C$, ktorá
začína vo vrchole $1$ a končí vo vrchole $x$ a množinu ciest $R$, ktoré
spĺňajú podmienky množín $A, B, C$, kde cesty z $R$ sú disjuktné s $p$ a navyše cesty
z $R$ a $p$ sú v súčte čo najkratšie.

Tento podproblém vieme jednoducho riešiť pomocou riešení menších podproblémov.
Vyskúšame všetky predchádzajúce vrcholy pre vrchol $x$ v prvej ceste a zároveň podľa jeho
príslušnosti do množiny $A, B, C$ si vyberieme vhodné zapojenie do druhej cesty.
Keďže nové hrany pridávame len poslednému vrcholu, vieme priamočiaro kontrolovať podmienku
disjuknosti ciest.

TODO obrazok


Tento algoritmus vieme ešte urýchliť jedným orezávaním. Veľkosť množiny
$B$ môže byť maximálne $\frac{2}{3}n$. V opačnom prípade by sme nevedeli tieto
cesty spojiť do kružnice (každý zo zvýšných vrcholov vie spojiť dva voľné konce
v množine $B$).

Teraz odhadneme zložitosť vyššie spomínaného algoritmu. Hlavným faktorom bude počet rôznych
podproblémov (vyriešenie podproblému nám trvá polynomiálny čas od $n$).

Nech $P(2k)$ je počet rôznych rozdelení $2k$ prvkov do dvojíc:
$$P(2k) = \frac{(2k)!}{k! 2^k}$$

Pre jednoduchosť dodefinujeme $P(2k+1) = P(2k)$.

%Nech $Z(m, n)$ je počet rôznych podproblémov pre vstup s $n$ vrcholmi, kde platí 
%$|A| + |B| + |C| = m$:
%$$Z(m, n) = m \binom{n}{m} \sum_{\substack{0 \leq k,\\ 2k \leq \min \left (m, \frac{2n}{3} \right )}} \binom{m}{2k} 2^{m-2k} P(2k)$$
%
%Tento výraz sme dostali z toho, že najprv vyberieme $m$ vrcholov z $n$. Z nich
%navyše vyberieme osobitný posledný vrchol.
%Potom tieto vrcholy musíme rozdeliť do množín $A, B, C$. Najprv vyberieme
%$2k$ vrcholov do množiny $B$, pre zvyšné vrcholy máme $2^{m-2k}$ možností
%pre rozdelenie do množín $A, C$. Navyše vrcholy v množine $B$ potrebujeme rozdeliť do dvojíc.
%
%Celkový počet podproblémov dostaneme ako:
%$$S(n) = \sum_{m=0}^m Z(m, n)$$

Celkový počet podproblémov odhadneme zhora. Máme najviac $4^n$ možností pre rozdelnie
vrcholov do množín $A, B, C$ a zvyšok. Jeden vrchol navyše vyberieme ako špeciálny.
Ešte potrebujeme spárovať množinu $B$ a na to máme najviac $P\left(\floor*{\frac{2}{3}n}\right)$
možností. A teda počet rôznych podproblémov môžeme odhadnúť ako:
$$S(n) \leq n 4^n P\left(\floor*{\frac{2}{3}n}\right)$$

Keď na $P(2k)$ použijeme Stirlingovu approximáciu dostávame:
$$P(2k) = \frac{\sqrt{4\pi k}\frac{(2k)^{2k}}{e^{2k}} (1 + o(1))}
{\sqrt{2\pi k}\frac{k^{k}}{e^{k}} 2^k (1 + o(1))} =
\frac{\sqrt{2} \cdot{}2^k n^k}{e^k} (1 + o(1)) =
\frac{2^k k!}{\sqrt{\pi n}} (1 + o(1)) $$

Po dosadení máme:
$$S(n) = \tilde{O}\left(2^{2.334n} \left(\frac{n}{3}\right)!\right)$$

Rovnaká (až na polynomiálny faktor) je aj časová zložitosť vyššie popisaného algoritmu.
