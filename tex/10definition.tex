V tejto kapitole formálne zadefinujeme problém dvoch obchodných cestujúcich
a jeho variácie, ktoré budeme riešiť.

\begin{definicia}[Problém dvoch obchodných cestujúcich]
Majme daný úplný ohodnotený graf $G = (V, E)$ s $n$ vrcholmi, pričom každá hrana
$e \in E$ má dĺžku $c(e)$. Cieľom je nájsť dve hranovo disjuktné hamiltonovské kružnice
$K_1 \subset E$, $K_2 \subset E$, tak aby súčet ich dĺžok bol minimálny.
Kružnice $K_1, K_2$ budeme volať aj TSP-cesty.
\end{definicia}

\begin{poznamka}
Cenu hranu spájajúcej vrcholy $u, v$ budeme označovať ako $c(u, v)$. 
\end{poznamka}

V niektorých prípadoch kladieme na dĺžky hrán ďalšie obmedzenia.
Jedným z obmedzení je trojuholníková nerovnosť:

\begin{definicia}{Trouholníková nerovnosť}
Hrany uhodnoteného grafu $G = (V, E)$ splňajú trouholníkovú nerovnosť, ak platí:
$$\forall u,v,w \in V: c(u,v) + c(v,w) \geq c(u,w)$$
\end{definicia}

Ešte silnejším obmedzením je požadovať, aby vrcholy grafu zodpovedali bodom v rovine, t.j.
každému vrcholu priradíme reálne čísla $x_v, y_v$ a vzdialenosť vrcholov $u, v$ bude:
$\sqrt{(x_u - x_v)^2 + (y_u - y_v)^2}$.


