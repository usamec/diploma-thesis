
%% Template by Michal Forisek


\documentclass[12pt,a4paper]{book}
\usepackage{slovak}
\usepackage[utf8]{inputenc}
\usepackage{a4wide}
\usepackage{tabularx}
\usepackage{amsfonts}
\usepackage{amssymb}
\usepackage{amsmath}
\usepackage{epsfig}
\usepackage{color}
\usepackage{mathrsfs}
\usepackage{verbatim}
\usepackage{hyperref}
\usepackage{subcaption}
\usepackage{float}
\usepackage{longtable}
\usepackage{listings}
\usepackage{graphicx}
\usepackage{mathtools}
\usepackage{geometry}
\usepackage{pdflscape}
\input{utility/makra.tex}

\linespread{1.3}

\begin{document}

% Obal
\thispagestyle{empty}
\begin{minipage}{0.25\textwidth}
\includegraphics[width=0.9\textwidth]{img/komlogo-new}
\end{minipage}
\begin{minipage}{0.69\textwidth}
\begin{center}
UNIVERZITA KOMENSKÉHO V BRATISLAVE \\
FAKULTA MATEMATIKY, FYZIKY A INFORMATIKY \\
\end{center}
\end{minipage}

\bigskip

\vfill
\begin{center}
\begin{minipage}{0.8\textwidth}
%\hrule
\bigskip\medskip
\centerline{\LARGE\sc Problém dvoch obchodných cestujúcich}
\end{minipage}
\end{center}
\vfill
{\bf 2013}

\hfill{\bf Vladimír Boža}
\eject % EOP i

%Titulna strana
\thispagestyle{empty}
\begin{minipage}{0.25\textwidth}
\includegraphics[width=0.9\textwidth]{img/komlogo-new}
\end{minipage}
\begin{minipage}{0.69\textwidth}
\begin{center}
UNIVERZITA KOMENSKÉHO V BRATISLAVE \\
FAKULTA MATEMATIKY, FYZIKY A INFORMATIKY \\
\end{center}
\end{minipage}

\vfill
\begin{center}
\begin{minipage}{0.8\textwidth}
%\hrule
\bigskip\medskip
\centerline{\LARGE\sc Problém dvoch obchodných cestujúcich}
\smallskip
\centerline{(diplomová práca)}
\bigskip
\bigskip
%\centerline{\large\sc Vladimír Boža}
\bigskip\bigskip
%\hrule
\end{minipage}
\end{center}
\vfill
\begin{tabular}{l l}
Študijný program: & Informatika \\
Študijný odbor: & 9.2.1 Informatika \\
Školiace pracovisko: & Katedra informatiky \\
Školiteľ: & RNDr. Michal Forišek PhD. \\
\end{tabular}
\vfill
{\bf Bratislava, 2013}

\hfill{\bf Vladimír Boža}
\eject % EOP i

\newgeometry{margin=1cm}
\thispagestyle{empty}
%\includegraphics[clip,trim=2.5cm 0 0 0,scale=1]{img/zadanie.PDF}
\includegraphics[clip,trim=1.5cm 5cm 0 0,scale=1]{img/zadanie.PDF}
\eject
\restoregeometry

\thispagestyle{empty}

\section*{Abstrakt}

V práci sa zaoberáme modifikáciou problému obchodného cestujúceho --
problémom dvoch obchodných cestujúcich, t.j. hľadáme dve hranovo
disjuktné hamiltonovské kružnice, ktorých súčet dĺžok je čo najmenší.
Pre riešenie tohoto problému sme pomocou dynamického programovania
zostrojili exaktný algoritmus, ktorého čas behu je výrazne lepší ako
čas behu triviálneho algoritmu. Ďalej sme modifikovali PTAS pre problém
obchodného cestujúceho v euklidovskej rovine, aby sme pomocou neho zostrojili PTAS pre
problém dvoch obchodných cestujúcich.
Nazáver ešte porovnávame praktickú výkonnosť heuristík využívajúcich
celočíselné lineárne programovanie. 

\medskip
{\bf Kľúčové slová:} obchodný cestujúci, PTAS, dynamické programovanie, celočíselné
lineárne programovanie
\eject

\thispagestyle{empty}

\section*{Abstract}

In this work we focus on modification of travelling salesman problem --
2-peripatetic salesman problem, where the aim is to find two disjoint
hamiltonian circles of minimum total cost.
We use dynamic programming to develop exact algorithm, which running time is
much smaller than running time of trivial algorithm. We modified PTAS for
travelling salesman problem in euclidean plane to develop PTAS for
2-peripatetic salesman problem. 
We finish our work with practical comparison of several heuristics
using integer linear programming.

\medskip
{\bf Key words:} travelling salesman, PTAS, dynamic programming,
integer linear programming
\eject

\thispagestyle{empty}
\tableofcontents
\thispagestyle{empty}

\setcounter{page}{0}
\chapter*{Úvod}
\addcontentsline{toc}{chapter}{Úvod}
\label{chapter:uvod}
Väčšina programovacích jazykov okrem základných príkazov obsahuje aj sadu knižníc,
ktorá implementuje ďalšie užitočné funkcionality. Pre nás sú zaujímavé hlavne knižnice, ktoré
implementujú základné algoritmy a kontajnery v danom jazyku (príkladom je Standard Template
Library - \cite{STL} z C++). Dôležitou vlastnosťou týchto algoritmov je ich schopnosť
fungovať nezávisle na type spracovávaných dát.

V kompilátoroch pre jazyk Pascal sa doteraz takéto knižnice nevyskytovali.
Jedným z hlavných dôvodov bola slabá alebo takmer žiadna podpora pre generics.
V roku 2008 vyšla verzia 2.2 kompilátora FreePascal, ktorá zaviedla chabú podporu pre
generics. My sme sa rozhodli niektoré základné kontejnery pre FreePascal pomocou
generics implementovať a pretlačiť ich medzi knižnice distribuované spolu s FreePascalom.

Členenie tejto práce je nasledovné. V prvej kapitole prezentujeme rozhranie
našej knižnice, spolu s niekoľkými jednoduchými príkladmi použitia. Táto
kapitola tvorí akýsi základný manuál. V druhej kapitole prezentujeme implementačné
detaily jednotlivých komponentov. 
V poslednej kapitole ukážeme niekoľko experimentálnych porovnaní efektívnosti
našej implementácie s implementáciou STL v C++.


\chapter{Definícia problému}
\label{chapter:definition}
V tejto kapitole formálne zadefinujeme problém dvoch obchodných cestujúcich
a jeho variácie, ktoré budeme riešiť.

\begin{definicia}[Problém dvoch obchodných cestujúcich]
Majme daný úplný ohodnotený graf $G = (V, E)$ s $n$ vrcholmi, pričom každá hrana
$e \in E$ má dĺžku $c(e)$. Cieľom je nájsť dve hranovo disjuktné kružnice $K_1 \subset E$,
$K_2 \subset E$, tak aby súčet ich dĺžok bol minimálny. Kružnice $K_1, K_2$ budeme volať aj
TSP-cesty.
\end{definicia}

V niektorých prípadoch kladieme na dĺžky hrán ďalšie obmedzenia.
Jedným z obmedzení je trojuholníková nerovnosť:

\begin{definicia}{Trouholníková nerovnosť}
Hrany uhodnoteného grafu $G = (V, E)$ splňajú trouholníkovú nerovnosť, ak platí:
$$\forall u,v,w \in V: c(u,v) + c(v,w) \geq c(u,w)$$
\end{definicia}

Ešte silnejším obmedzením je požadovať, aby vrcholy grafu zodpovedali bodom v rovine.

TODO: definicia



\chapter{Doterajšie výsledky}
\label{chapter:previous}
Problém dvoch obchodných cestujúcich je zjavne NP-úplný, ako to dokázal
už \cite{krarup}, keď tento problém zadefinoval.

V nasledujúcich riadkoch uvedieme niekoľko doterajších výsledkoch, ktoré sa vyskytli
pri riešení tohoto problému.

\section{Aproximačné algoritmy}

Dá sa pomerne ľahko ukázať, že pre problém dvoch obchodných cestujúcich vo všeobecných grafoch
neexistuje aproximačný algoritmus s konštantným aproximačným pomerom. Dôkaz je v prakticky
rovnaký ako pre problém obchodného cestujúceho.

V prípade, že graf spĺňa trojuholníkovu nerovnosť sa už dajú nájsť approximačné algoritmy.
Dá sa napríklad ukázať, že ak má graf dostatočne veľa vrcholov, tak ku každej TSP ceste
existuje TSP cesta, ktorá je s ňou disjuktná a má dĺžku najviac rovnú $(2+\epsilon)$-násobku
dĺžky pôvodnej cesty.
Použítím $1.5$-aproximácie pre problém obchodného cestujúceho (\cite{christofides}) vieme
dostať $(9/4 + \epsilon)$-aproximačný algoritmus. 

Dá dosiahnúť aj $2$-aproximačný algoritmus, ako to je ukázané v \cite{apx2}. Hlavnou
ideou je zostrojiť najprv dve disjuktné kostry s minimálnou cenou pomocou algoritmu
uvedeného v \cite{spanning2} a následne pomerne technickou konštrukciou zostrojiť
s použitím týchto kostier dve disjunktné hamiltonovské kružnice.

V prípade, že ceny hrán môžu byť len z množiny $\{1,2\}$ existuje $11/9$-aproximačný
algoritmus (\cite{gimadi}). V tomto článku sa nachádza navyše niekoľko výsledkov pre 
variáciu problému, kde cenu ciest maximalizujeme.

\section{Riešenia pomocou celočíselného lineárneho programovania}

\subsection{Riešenie problému obchodného cestujúceho pomocou celočíselného
lineárneho programovania}

Najprv zhrnieme niekoľko výsledkov o riešení problému obchodného cestujúceho pomocou
celočíselného lineárneho programovania, ktoré sa nachádzajú v \cite{tspsurvey}.

Problém obchodného cestujúceho sa dá priamočiaro popísať nasledujúcim celočíselným lineárnym
programom:

$$\min \sum_{e \in E} c(e) x_e$$ 

Za predpokladov:
\begin{subequations}
\begin{equation}\forall e \in E: x_e \in \{0, 1\} \label{eq:enum}\end{equation}
\begin{equation}\forall v \in V: \sum_{e \in \delta(v)} x_e = 2 \label{eq:vertexsum1} \end{equation}
\begin{equation}\forall S \subset V, S \neq \emptyset: \sum_{e \in \delta(S)} x_e \geq 2 \label{eq:subtour1}\end{equation}
\end{subequations}

Kde $\delta(v)$ je množina všetkých hrán, ktoré majú koniec vo vrchole $v$ a
$\delta(S)$ je množina všetkých hrán, ktorá majú jeden koniec v $S$ a druhý v $V \setminus S$.

Podmienky \eqref{eq:vertexsum1} zabezpečujú, aby z každého vrchola výchádzali práve
dve hrany. Podmienky \eqref{eq:subtour1} zabezpečujú, aby sme mali iba jeden cyklus.
Táto formulácia má ale exponenciálne veľa podmienok a na prvý pohľad je neužitočná.

Existuje formulácia, ktorá má iba polynomiálne veľa podmienok. V tomto prípade
ale použijeme orientované hrany:

$$\min \sum_{i\neq j} c(i,j) x_{ij}$$

Za predpokladov:
\begin{subequations}
\begin{equation}\forall j \neq i_0: \sum_{i=1}^n x_{ij} = 1\end{equation}
\begin{equation}\forall i \neq i_0: \sum_{j=1}^n x_{ij} = 1\end{equation}
\begin{equation}\forall i,j, i \neq i_0 \vee j \neq j_0: u_i + u_j + nx_{ij} \leq n-1\end{equation}
\end{subequations}

Cieľom premenných $u_i$ je zaviesť očíslovanie vrcholov na kružnici. Tento program
má síce len polynomiálne veľa podmienok, ale ukazuje sa, že nie je ľahko riešiteľné
pomocou bežných nástrojov na riešenie celočíselných lineárnych programov.

Ukazuje sa, že pomerne účinný algoritmus je nasledovný:

\begin{enumerate}
\item Vytvor lineárny program, ktorá obsahuje iba podmienky \eqref{eq:enum}, 
\eqref{eq:vertexsum1}.
\item Najdi riešenie daného lineárneho programu.
\item Ak je riešenie lineárneho programu iba jedna kružnica, tak máme dobré riešenie a skonči.
\item Ináč nájdi porušené podmienky \eqref{eq:subtour1} a pridaj ich do lineárneho
programu. A pokračuj krokom 2.
\end{enumerate}

Tomuto postupu sa zvykne hovoriť aj "subtour elimination".

\subsection{Riešenie problému dvoch obchodných cestujúcich pomocou celočíselného
lineárneho programovania}

V nasledujúcej časti zhrnieme výsledky z \cite{duchenne}.
Priamym rozšírením postupu pre obchodného cestujúceho na dvoch cestujúcich
dostaneme nasledovný program:

$$\min \sum_{e \in E} c(e) x_e + \sum_{e \in E} c(e) y_e$$ 

Za predpokladov:
\begin{subequations}
\begin{equation}\forall e \in E: x_e \in \{0, 1\}\end{equation}
\begin{equation}\forall e \in E: y_e \in \{0, 1\}\end{equation}
\begin{equation}\forall e \in E: x_e + y_e \leq 1\end{equation}
\begin{equation}\forall v \in V: \sum_{e \in \delta(v)} x_e = 2\end{equation}
\begin{equation}\forall v \in V: \sum_{e \in \delta(v)} y_e = 2\end{equation}
\begin{equation}\forall S \subset V, S \neq \emptyset: \sum_{e \in \delta(S)} x_e \geq 2
\label{eq:subtour2}\end{equation}
\begin{equation}\forall S \subset V, S \neq \emptyset: \sum_{e \in \delta(S)} y_e \geq 2
\label{eq:subtour3}\end{equation}
\end{subequations}

Spôsob riešenie je podobný ako pri riešení problému obchodného cestujúceho -- najprv
nepoužijeme žiadne z podmienok \eqref{eq:subtour2}, \eqref{eq:subtour3} a postupne
pridáme len porušené podmienky. V \cite{duchenne} sa tento algoritmus nazýva "3-index".

Ukazuje sa, že tento algoritmus je pomerne pomalý a potrebuje veľa iterácií riešenia
lineárneho programu. Pri praktickom testovaní na náhodných euklidovských grafoch
je tento algoritmus úspešný maximálne pre grafy s $50$ vrcholmi. Hlavným problémom
tohoto algoritmu je, že pomerne často nastáva taký prípad, keď len prehadzujeme hrany z $x$ 
do $y$ a naopak. 

\smallskip

Ako lepší sa ukazuje algoritmus s nasledujúcou myšlienkou: Namiesto hľadania dvoch
disjuktných hamiltonovských kružníc budeme hľadať najlacnejší 4-regulárny 4-súvislý podgraf.
Následne ak sa tento podgraf dá rozložiť na dve hamiltonovské kružnice, tak máme riešenie.
Ináč tento podgraf zakážeme a budeme hľadať najlacnejší nezakázaný podgraf.
Iný pohľad na tento algoritmus je taký, že v lineárnom programe zlúčíne premenné $x_e$ a $y_e$ 
do jednej, čiže aktuálny program, ktorý budeme nazývať 2-index vyzerá takto:

$$\min \sum_{e \in E} c(e) x_e$$ 

Za predpokladov:
\begin{subequations}
\begin{equation}\forall e \in E: x_e \in \{0, 1, 2\}\end{equation}
\begin{equation}\forall v \in V: \sum_{e \in \delta(v)} x_e = 4\end{equation}
\begin{equation}\forall S \subset V, S \neq \emptyset: \sum_{e \in \delta(S)} x_e \geq 4
\label{eq:subtour4}
\end{equation}
\end{subequations}

Aktuálne už hľadanie porušených podmienok \eqref{eq:subtour4} nie je také jednoduché.
Nestačí už len hľadať komponenty vzniknutého grafu potrebujeme hľadať rezy, ktoré
majú veľkosť menšiu ako 4. Na toto môžeme napríklad použiť algoritmus z \cite{stoer}. 

Väčším problémom je zistenie, či sa dá 4-regulárný 4-súvislý podgraf rozdeliť na dve
disjuktné hamiltonovské kružnice. Tento problém je NP-úplný ako ukázal \cite{hamdecomp}.
V \cite{duchenne} na riešenie tohoto problému používajú upravený 3-index algoritmus.
Takýto 2-index algoritmus funguje pre náhodné euklidovské grafy, ktoré majú menej ako $200$
vrcholov.


\chapter{Pomalé metódy riešenia}
\label{chapter:slow}
V tejto kapitole sa budeme venovať pomalým metódam riešenia problému
dvoch obchodných cestujúcich.

\section{Skúšanie všetkých možností}

V grafe s $n$ vrcholmi sa nachádza $n!$ rôznych kružníc. Pokiaľ hľadáme
dve disjuktné kružnice, tak rôzných dvojíc kružníc je $\tilde{O}((n!)^2)$.
Každú z týchto možností vieme overiť v čase $O(n)$. Celkový čas algoritmu
je teda $\tilde{O}((n!)^2)$.

\section{Dynamické programovanie pre problém obchodného cestujúceho}

Problém obchodného cestujúceho sa dá riešiť pomocou dynamického
programovania v čase $O(2^n n^2)$ (\cite{Held}). Hlavnou myšlienkou
tohoto algoritmu je riešiť podproblémy tvaru: Nájdite najkratšiu cestu, ktorá
začína vo vrchole 1, prechádza cez množinu vrcholov $X \subseteq V$ a končí vo vrchole
$x \in X$.

Tento algoritmus vieme využiť pomerne priamočiaro. Pre každú z $n!$ možných kružníc
v čase $\tilde{O}(2^n)$ nájdeme najkratšiu druhú kružnicu. Takto dostáneme
algoritmus, ktorý beží v čase $\tilde{O}(2^n n!)$. 



\chapter{Aproximačné algoritmy}
\label{chapter:ptas}
Zamerajme sa teraz na inštancie, kde vrcholy grafu reprezentujú body v rovine a hrany
vzdialenosti medzi nimi. V prípade obyčajného problému obchodného cestujúceho
existuje PTAS (polynomial time approximation scheme) (\cite{Arora}), t.j. algoritmus, ktorý
pre ľubovoľný aproximačný pomer beží v polynomiálnom čase od veľkosti vstupu.

My v nasledujúcej časti použijeme myšlienky z tohoto algoritmu a ukážeme ako spraviť PTAS pre
problém dvoch obchodných cestujúcich.

Hlavnou myšlienkou algoritmu pomocou techniky rozdeľuj a panuj vybudovať nad vstupom
randomizovaný quad-tree. Následne ukážeme, že pre $(1 + 1/c)$-aproximačný algoritmus stačí, aby
TSP-cesty pretínali hrany delenia iba $O(c)$ krát. Navyše tieto prieniky sa môžu
udiať iba v niektorých špeciálnych bodoch.

\section{Rekurzívne delenie}

%Rovinu v ktorej ležia vstupné body budeme rekurzívne deliť na menšie časti.
%Najprv rovinu rozdelíme vodorovnou a zvislou čiarou na štyri časti, každú z nich na ďalšie štyri, ...
%Takto by sme dostali tzv. quadtree. Na správnu fukčnosť algoritmu potrebujeme ale tzv.
%posunutý quadtree, kde deliace čiary neprechádzajú stredom, ale sú vyberané náhodne.
%

Pre začiatok predpokladajme, že všetky body majú celočíselné súradnice a vzdialenosť
medzi nimi je aspoň $8$. Neskôr ukážeme, že každý vstup vieme takto upraviť a ak budeme mať PTAS
pre upravený vstup, tak budeme mať PTAS aj pre pôvodný vstup.

Nech všetky body ležia vo vnútri štvorca $S$ so stranou $L$, ktorého ľavý dolný roh
má súradnice $[0,0]$. Navyše bez ujmy na všeobecnosti nech je $L$ mocnina dvojky.

{\bf Rekurzívnym delením} štvorca $S$ nazývame jeho rekurzívne delenie na menšie štvorce tak, že
každy štvorec rozdelíme {\bf deliacimi čiarami} na štyri rovnaké štvorce.
Deliť prestaneme, keď je strana štvorca $\leq 1$. Toto delenie predstavuje
$4$-árny strom, kde synovia každé ho štvorca sú štyri menšie štvorce. Hĺbka takéhoto stromu je
najviac $O(\lg L)$. 
Štvorcom delenia intuitívne priradíme úrovne. Štvorec $S$ má úroveň $0$, jeho synovia úroveň $1$,
atď. Deliaca čiara má úroveň $i$ vtedy, keď obsahuje hranu nejakého štvorca úrovne $i$, ale žiadneho
úrovne väčšej ako $i$.

{\bf Quadtree} definujeme podobne 
až na to, že rekurzívne delenie skončí vtedy, keď v danom štvorci je najviac jeden bod. Keďže každý
list quadtree obsahuje bod zo vstupu, alebo je súrodenec takého listu, tak quadtree má maximálne
$O(n)$ listov a teda dokopy najviac $O(n \lg L)$ štvorcov.

\smallskip

Teraz popíšeme posunuté delenie. Nech $a, b$ sú celé čísla a platí $0 \leq a, b < L$.
$(a,b)$-posunutie rekurzívneho delenia definujeme tak, že $x$-, $y-$ súradnice každej deliacej čiary
zvýšime o $a$ resp. $b$ a následne spočítame ich zvyšok po delení číslom $L$ (vstupnými bodmi
nehýbeme). Takisto posunieme aj okrajové čiary štvorca (začnú tvoriť deliace čiary). Pôvodný štvorec
pokladáme za zacyklený, teda okrajové oblasti pri posunutom delení tvoria súvislú oblasť ako na
obrázku.

TODO obrázok

Z posunutého rekurzívneho delenia vieme dodefinovať quadtree s posunom tak, že nebudeme deliť
štvorce, ktoré obsahujú najviac jeden bod.

\section{Portály a $(m,r)$-ľahké cesty}

Nás algoritmus dovolí TSP-cestám pretínať čiary delenia iba v niekoľkých
predpísaných bodoch, tzv. {\bf portáloch}. Toto je na prvý pohľad trochu kontraintuitívne.
My ale dovolíme cestujúcim cestovať medzi bodmi po ľubovoľnej čiare, teda nie nutne najkratšej
ceste medzi dvoma bodmi (viď. obrázok nižšie). Stále ale zachovávame disjuktnosť ciest, t.j. keď
jeden cestujúci ide z vrchola $x$ do vrcholu $y$, tak druhý nemôže ísť z $x$ do $y$ ani z $y$ do
$x$ (nech ide ľubovoľne zakrivenou cestou). 

\begin{definicia}
Nech $m$ je párne kladné celé číslo. $m$-regulárna množina portálov pre posunuté rekurzívne
delenie je množina bodov taká, že každá deliacia čiara úrovne $i$ obsahuje portály vo
vzdialenostiach $\frac{L}{m2^i}$. 
\end{definicia}

TODO obrázok

\begin{definicia}
TSP-cesta sa nazýva $(m,r)$-ľahká ak pretína čiary rekurzívneho delenia iba v $m$-regulárnej
množine portálov a navyše každú hranu štvorca pretína najviac $r$-krát.
\end{definicia}




\chapter{Heuristiky}
\label{chapter:heuristiky}
TODO poriadny nazov kapitoly

V tejto časti popíšeme niekoľko algoritmov, ktoré sú pomerne úspešné
na praktických inštanciách problému, ale väčšinou je ich teoretická
analýza pomerne ťažká. Popíšeme niekoľko heuristík, ktoré v pomerne krátkom čase
dávajú riešenie, ktoré je väčšinou blížko optimálneho. Aby sme vedeli odhadnúť
kvalitu týchto heuristík, najprv ale ukážeme niekoľko algoritmov na hľadanie dolného odhadu
veľkosti riešenia. Následne ešte vylepšíme algoritmus používajúci
celočíselné lineárne programovanie od \cite{duchenne}, aby
sme dokázali optimálne riešiť inštancie, ktoré majú najviac 400 vrcholov.

\section{Dolné odhady veľkosti riešenia}

\subsection{Odhad pomocou dvoch disjuktných kostier}

Pomerne priamočiarym dolným odhadom veľkosti riešenia je veľkosť
dvoch nakratších disjuktných kostier. My to jemne vylepšíme budeme používať
tzv. 1-kostry.

\begin{definicia}
1-kostra v grafe $G=(V, E)$ je zjednotenie kostry na vrcholoch $V \setminus {1}$
a dvoch hrán susedných s vrcholom $1$. 
\end{definicia}

\begin{lema}
Veľkosť dvoch najlacnejších disjuktných 1-kostier je najviac tak veľká ako dĺžka
riešenia problému dvoch obchodných cestujúcich.
\end{lema}

\begin{dokaz}
Každá hamiltonovská kružnica je 1-kostra. Riešenie problému dvoch obchodných cestujúcich
sú teda dve disjuktné 1-kostry.
\end{dokaz}

Dve najkratšie disjuktné 1-kostry môžeme hľadať pomocou algoritmu popísaného v
\cite{spanning2} -- nájdeme najkratšie disjuktné kostry a k vrcholu $1$ ešte pridáme
dve najkratšie hrany.

Tento odhad sa dá vylepšiť pomocou techniky opísanej v \cite{heldtsp} a \cite{lower1}.
Jej princíp je nasledovný. Každému vrcholu prirádíme potenciál 
$p_i \in \mathbb{R}$.
A zavedenieme nové dĺžky hrán:
$$d(u, v) = c(u, v) + p_u + p_v$$

\begin{lema}
Majme potenciály $p_1, p_2, \dots, p_n$. Riešenie problému dvoch obchodných
cestujúcich pri dĺžkach $d(u,v)$ definovaných vyššie je rovnaké ako riešenie
problému dvoch obchodných cestujúcich pri dĺžkach $c(u,v)$.
\end{lema}

\begin{dokaz}
Majme hamiltonovskú kružnicu $v_1, v_2, \dots, v_n$ s dĺžkou $C = c(v_1, v_2) + c(v_2, v_3) +
\dots + c(v_n, v_1)$. Pokiaľ zavedieme dĺžky hrán $d(u,v)$ dostaneme cenu:
$$C^{'} = c(v_1, v_2) + p_{v_1} + p_{v_2} + c(v_2, v_3) + p_{v_2} + p_{v_3} + \dots +
c(v_n, v_1) + p_{v_n} + p_{v_1}$$
$$C^{'} = C + 2 \sum_{v \in V} p_v$$

To znamená, že ku dĺžke každej hamiltonovskej kružnice prirátame rovnaké číslo
a teda najlepšie riešenie pri upravených dĺžkach je rovnaké ako najlepšie
riešenie pri pôvodných dĺžkach.
\end{dokaz}

Podstatné je, že zmenou potenciálov vieme zmeniť najlacnejšie disjuktné 1-kostry. 
Našim cieľom bude nájsť také potenciály, aby rozdiel medzi dĺžkou najlacnejších
disjuktných 1-kostier a dĺžkou riešenia problému dvoch obchodných cestujúcich
bol čo najmenší. Nech $C$ je dĺžka riešenia problému dvoch obchodných cestujúcich
pri nulových potenciáloch. Nech $D$ je dĺžka dvoch najlacnejších disjuktných 1-kostier
pri potenciáloch $p_1, \dots, p_n$. Potom spomínaný rozdiel vieme vyjadriť ako:
$$C + 2 \sum_{v \in V} p_v - D$$

V našom prípade je $C$ nám neznáma konštanta a teda stačí minimalizovať výraz:
$$2 \sum_{v \in V} p_v - D$$

Vhodné potenciály vieme nájsť napríklad pomocou subgradientovej optimalizácie.
Dá sa totiť ukázať, že ak $d_v$ je súčet stupňov vrchola $v$ v obidvoch kostrách, tak
vektor so zložkami $g_v = 4 - d_v$ je subgradient pre vyššie spomínaný optimalizačný
problém.

Na začiatku položíme všetky potenciály ako $p^{(0)}_i = 0$.


%\chapter{Rozhranie knižnice}
%\label{chapter:rozhranie}
%\input{tex/10reference.tex}
%\input{tex/11containers.tex}
%\input{tex/12ordered_containers.tex}
%\input{tex/13unordered_containers.tex}
%\input{tex/14array_utils.tex}
%
%\chapter{Implementácia}
%\label{chapter:implementation}
%\input{tex/21sequence_implementation.tex}
%\input{tex/22ordered_implementation.tex}
%\input{tex/23unordered_implementation.tex}
%\input{tex/24sorting.tex}
%\input{tex/25testing.tex}
%
%\chapter{Praktické testy}
%\label{chapter:practical}
%\input{tex/3practical.tex}
%\input{tex/31sorting.tex}
%\input{tex/32lcs.tex}
%\input{tex/33asfalt.tex}
%\input{tex/34basnik.tex}
%\input{tex/35conclusion.tex}

\chapter*{Záver}
\addcontentsline{toc}{chapter}{Záver}
\label{chapter:fin}
V tejto práci sme zhrnuli a navrhli niekoľko nových prístupov
pre riešenie problému dvoch obchodných cestujúcich.
Naše hlavné prínosy sú nasledovné:
\begin{itemize}
\item Navrhli sme exaktný algoritmus, ktorého časová zložitosť je oveľa
lepšia ako zložitosť triviálneho algoritmu.
\item Ukázali sme ako urobiť PTAS pre problém dvoch obchodných cestújich.
Navyše jeho zložitosť nie je horšia ako zložitosť PTASu pre problém obchodného
cestujúceho.
\item Navrhli sme niekoľko heuristík a porovnali ich s doterajšími.
Kým doterajšie heuristiky zvládali inštancie s maximálnou veľkosťou 280, naše
heuristiky zvládajú inštancie s maximálnou veľkosťou 442.
\end{itemize}

V spomínaných výsledkoch je stále niekoľko miest na zlepšenie.
Pri exaktnom algoritme je zaujímavé zistiť, či obmedzenie na vstupný graf
(napr. trojuholníková nerovnosť) neumožní návrh ešte rýchlejšieho algoritmu.
Pri heuristikách by bolo veľmi nápomocné, ak by sa podarilo nájsť postup
ako znížiť počet zistovaní toho, či je 4-súvislý 4-regulárny graf rozložiteľný
na dve hamiltonovské kružnice.


%\chapter{Prehľad štandardných algoritmov - obsolete, nema tu co robit} 
%\label{chapter:salg}
%\input{tex/x1sorting.tex}
%\input{tex/x2containers.tex}
%\input{tex/x3sets.tex}

%\backmatter fixme: preco to tu nefunguje? asi chyba nejaky package
%\listoffigures
%\listoftables

\bibliographystyle{alpha}
\bibliography{literatura}



\end{document}
