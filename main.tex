
%% Template by Michal Forisek


\documentclass[12pt,a4paper]{report}
\usepackage{slovak}
\usepackage[utf8]{inputenc}
\usepackage{a4wide}
\usepackage{tabularx}
\usepackage{amsfonts}
\usepackage{amssymb}
\usepackage{amsmath}
\usepackage{epsfig}
\usepackage{color}
\usepackage{mathrsfs}
\usepackage{verbatim}
\usepackage{hyperref}
\usepackage{subfigure}
\usepackage{float}
\usepackage{longtable}
\usepackage{listings}
\usepackage{graphicx}
\input{utility/makra.tex}

\begin{document}

% Obal
\thispagestyle{empty}
\begin{minipage}{0.25\textwidth}
\includegraphics[width=0.9\textwidth]{img/komlogo-new}
\end{minipage}
\begin{minipage}{0.69\textwidth}
\begin{center}
UNIVERZITA KOMENSKÉHO V BRATISLAVE \\
FAKULTA MATEMATIKY, FYZIKY A INFORMATIKY \\
\end{center}
\end{minipage}

\bigskip
3dc8ac68-0ab2-4817-b352-48b4038e9b46

\vfill
\begin{center}
\begin{minipage}{0.8\textwidth}
%\hrule
\bigskip\medskip
\centerline{\LARGE\sc Knižnica štandardných algoritmov}
\smallskip
\centerline{\LARGE\sc pre kompilátor FreePascal}
\end{minipage}
\end{center}
\vfill
{\bf 2011}

\hfill{\bf Vladimír Boža}
\eject % EOP i

%Titulna strana
\thispagestyle{empty}
\begin{minipage}{0.25\textwidth}
\includegraphics[width=0.9\textwidth]{img/komlogo-new}
\end{minipage}
\begin{minipage}{0.69\textwidth}
\begin{center}
UNIVERZITA KOMENSKÉHO V BRATISLAVE \\
FAKULTA MATEMATIKY, FYZIKY A INFORMATIKY \\
\end{center}
\end{minipage}

\vfill
\begin{center}
\begin{minipage}{0.8\textwidth}
%\hrule
\bigskip\medskip
\centerline{\LARGE\sc Knižnica štandardných algoritmov}
\smallskip
\centerline{\LARGE\sc pre kompilátor FreePascal}
\smallskip
\centerline{(bakalárska práca)}
\bigskip
\bigskip
%\centerline{\large\sc Vladimír Boža}
\bigskip\bigskip
%\hrule
\end{minipage}
\end{center}
\vfill
\begin{tabular}{l l}
Študijný program: & Informatika \\
Študijný odbor: & 9.2.1 Informatika \\
Školiace pracovisko: & Katedra informatiky \\
Školiteľ: & RNDr. Michal Forišek PhD. \\
\end{tabular}
\vfill
{\bf Bratislava, 2011}

\hfill{\bf Vladimír Boža}
\eject % EOP i

%\thispagestyle{empty}
%\includegraphics[scale=0.7]{zad.pdf}
%\eject

\thispagestyle{empty}

\section*{Abstrakt}

Sada štandardných algoritmov a kontajnerov je dôležitosťou súčasťou
skoro každého programovacieho jazyka. V tejto práci implementujeme najdôležitejšie
z nich pre kompilátor FreePascal, v ktorom sa doteraz nenachádzali.

Výsledkom práce je jedna ucelená knižnica priložená k práci a ktorá
bude tiež v dohľadnej dobe dostupná spolu s FreePascalom.

\medskip
{\bf Kľúčové slová:} FreePascal, generics, algoritmy a dátové štruktúry
\eject

\thispagestyle{empty}

\section*{Abstract}

Library implementing standard algorithms and containers is important part
of almost every programming language. We implemented most common
algorithms and container for FreePascal compiler, which did not have them yet.

Result of this work is library which is distributed with this work
and should be part of FreePascal compiler soon.

\medskip
{\bf Key words:} FreePascal, generics, algorithms and data structures
\eject

\thispagestyle{empty}
\tableofcontents
\thispagestyle{empty}

\setcounter{page}{0}
\chapter*{Úvod}
\addcontentsline{toc}{chapter}{Úvod}
\label{chapter:uvod}
Väčšina programovacích jazykov okrem základných príkazov obsahuje aj sadu knižníc,
ktorá implementuje ďalšie užitočné funkcionality. Pre nás sú zaujímavé hlavne knižnice, ktoré
implementujú základné algoritmy a kontajnery v danom jazyku (príkladom je Standard Template
Library - \cite{STL} z C++). Dôležitou vlastnosťou týchto algoritmov je ich schopnosť
fungovať nezávisle na type spracovávaných dát.

V kompilátoroch pre jazyk Pascal sa doteraz takéto knižnice nevyskytovali.
Jedným z hlavných dôvodov bola slabá alebo takmer žiadna podpora pre generics.
V roku 2008 vyšla verzia 2.2 kompilátora FreePascal, ktorá zaviedla chabú podporu pre
generics. My sme sa rozhodli niektoré základné kontejnery pre FreePascal pomocou
generics implementovať a pretlačiť ich medzi knižnice distribuované spolu s FreePascalom.

Členenie tejto práce je nasledovné. V prvej kapitole prezentujeme rozhranie
našej knižnice, spolu s niekoľkými jednoduchými príkladmi použitia. Táto
kapitola tvorí akýsi základný manuál. V druhej kapitole prezentujeme implementačné
detaily jednotlivých komponentov. 
V poslednej kapitole ukážeme niekoľko experimentálnych porovnaní efektívnosti
našej implementácie s implementáciou STL v C++.


%\chapter{Rozhranie knižnice}
%\label{chapter:rozhranie}
%\input{tex/10reference.tex}
%\input{tex/11containers.tex}
%\input{tex/12ordered_containers.tex}
%\input{tex/13unordered_containers.tex}
%\input{tex/14array_utils.tex}
%
%\chapter{Implementácia}
%\label{chapter:implementation}
%\input{tex/21sequence_implementation.tex}
%\input{tex/22ordered_implementation.tex}
%\input{tex/23unordered_implementation.tex}
%\input{tex/24sorting.tex}
%\input{tex/25testing.tex}
%
%\chapter{Praktické testy}
%\label{chapter:practical}
%\input{tex/3practical.tex}
%\input{tex/31sorting.tex}
%\input{tex/32lcs.tex}
%\input{tex/33asfalt.tex}
%\input{tex/34basnik.tex}
%\input{tex/35conclusion.tex}

\chapter*{Záver}
\addcontentsline{toc}{chapter}{Záver}
\label{chapter:fin}
V tejto práci sme zhrnuli a navrhli niekoľko nových prístupov
pre riešenie problému dvoch obchodných cestujúcich.
Naše hlavné prínosy sú nasledovné:
\begin{itemize}
\item Navrhli sme exaktný algoritmus, ktorého časová zložitosť je oveľa
lepšia ako zložitosť triviálneho algoritmu.
\item Ukázali sme ako urobiť PTAS pre problém dvoch obchodných cestújich.
Navyše jeho zložitosť nie je horšia ako zložitosť PTASu pre problém obchodného
cestujúceho.
\item Navrhli sme niekoľko heuristík a porovnali ich s doterajšími.
Kým doterajšie heuristiky zvládali inštancie s maximálnou veľkosťou 280, naše
heuristiky zvládajú inštancie s maximálnou veľkosťou 442.
\end{itemize}

V spomínaných výsledkoch je stále niekoľko miest na zlepšenie.
Pri exaktnom algoritme je zaujímavé zistiť, či obmedzenie na vstupný graf
(napr. trojuholníková nerovnosť) neumožní návrh ešte rýchlejšieho algoritmu.
Pri heuristikách by bolo veľmi nápomocné, ak by sa podarilo nájsť postup
ako znížiť počet zistovaní toho, či je 4-súvislý 4-regulárny graf rozložiteľný
na dve hamiltonovské kružnice.


%\chapter{Prehľad štandardných algoritmov - obsolete, nema tu co robit} 
%\label{chapter:salg}
%\input{tex/x1sorting.tex}
%\input{tex/x2containers.tex}
%\input{tex/x3sets.tex}

%\backmatter fixme: preco to tu nefunguje? asi chyba nejaky package
%\listoffigures
%\listoftables

\bibliographystyle{alpha}
\bibliography{literatura}



\end{document}
